\documentclass[a4paper,12pt, parskip=half-]{scrartcl}

\usepackage{lmodern}
\usepackage[latin1, UTF8]{inputenc}
\usepackage[english]{babel}
\usepackage[T1]{fontenc}
\usepackage{amsmath}

% Define Fonts
\usepackage{helvet} %Arial
\renewcommand{\familydefault}{\sfdefault} %Arial
%\usepackage{mathptmx} %Times New Roman

\usepackage{graphicx}
\usepackage{graphics}
\usepackage[a4paper, top=2.6cm, bottom=2.6cm, right=2.7cm, left=2.7cm]{geometry}
\usepackage[onehalfspacing]{setspace}
\usepackage{microtype}
\usepackage{breakcites}
\usepackage{booktabs}
\usepackage{tabularx}
\usepackage{multirow} 
\pagenumbering{gobble}
\usepackage[absolute]{textpos}
\usepackage{wrapfig}
\usepackage{float}
\usepackage{placeins}
\usepackage{afterpage}
\usepackage{enumitem}
\usepackage{lscape}
\usepackage{caption}
\usepackage{subcaption}
\usepackage{listings}
\usepackage{multirow}
\usepackage{abstract}
\usepackage{tikz}
\setkomafont{sectioning}{\bfseries} 
\usepackage{rotating}
\usepackage{color}													
\usepackage{colortbl}
\usepackage{pdfpages}
\usepackage{listings}

\makeatletter
\newcommand{\MSonehalfspacing}{%
  \setstretch{1.44}%  default
  \ifcase \@ptsize \relax % 10pt
    \setstretch {1.448}%
  \or % 11pt
    \setstretch {1.399}%
  \or % 12pt
    \setstretch {1.433}%
  \fi
}
\makeatother
\MSonehalfspacing

\usepackage[authoryear,round]{natbib}
\bibliographystyle{apalike}

\deffootnote{0.6cm}{1em}{\makebox[0.6cm][l]{\thefootnotemark}}
\setkomafont{footnote}{\fontsize{10pt}{10pt}\selectfont}

\usepackage{fancyhdr}
\pagestyle{plain}

\definecolor{mygreen}{rgb}{0,0.6,0}
\definecolor{mygray}{rgb}{0.5,0.5,0.5}
\definecolor{mymauve}{rgb}{0.58,0,0.82}

\begin{document}

\textbf{Question 4: Summary of De Nardi (2004)}\\

Since the distribution of wealth is much more concentrated than the distribution of (labour) earnings, De Nardi (2004) estimates a quantitative, general equilibrium, incomplete markets, overlapping-generations model to match this observation from the data. In doing so, the author links parents and children by accidental as well as voluntary bequests - but not by inter vivo transfers - and allows children to inherit some of their parents' productivity. Further, she calibrates the model to match important features of, first, the data of the U.S. and, second, the data of Sweden. Both countries have different wealth distribitions even though their Gini coefficients are relatively similar. This paper's framework makes it possible for intergenerational links to induce saving behaviour that generates a more concentrated distribution of wealth than that of earnings due to allowing two different saving motives: self-insurance against labour earnings shocks and life-span risk, i.e. saving for retirement, which both can be seen as precautionary motives, and the preference to leave bequests to their offspring, which is an altruistic motive. 
\\
De Nardi finds that saving for precautionary purposes and saving for retirement are the primary factors for wealth accumulation at the lower tail of the distribution, while saving to leave bequests significantly affects the shape of the upper tail.






\end{document}





